\documentclass{TDP019mall}
\usepackage{graphicx}
\graphicspath{{./bilder/}}


\newcommand{\version}{Version 1.0}
\author{Ahmed Sikh , \url{ahmsi881@student.liu.se}\\
Sayed Ismail Safwat, \url{saysa289@student.liu.se}}

\title{Språkdokumentation}
\date{\today}
\rhead{Ahmed Sikh\\
Sayed Ismail Safwat}


\begin{document}
\begin{center}
\includegraphics[width = 125mm]{logo}
\end{center}

\projectpage
\tableofcontents
\newpage
\section{Revisionshistorik}
\begin{table}[!h]
\begin{tabularx}{\linewidth}{|l|X|l|}
\hline
Ver. & Revisionsbeskrivning & Datum \\\hline
1.0 & Första version av Språkdokumentation & 210430 \\\hline
\end{tabularx}
\end{table}



\section{Inledning}
Detta är ett projekt på IP-programmet som är skapat under den andra terminen vid Linköpings universitet i kursen TDP019 Projekt: datorspråk.

\subsection{Syfte}
Syftet med denna kursen var att visa vilka komponenter ett språk består av och hur ett nytt programmeringsspråk byggs upp med de där komponenterna. 

\subsection{Introduktion}
I det här språket har tagits inspiration för det mesta från Ruby språket. ETL är utvecklats för en nybörjare användare och är skrivet i ett sätt som 
liknar skriftligt engelska vilket gör det möjligt för språkets läsbarhet.

\subsection{Målgrupp}
ETL (Easy To Learn) språket skall passa de nybörjare som har inga tidigare förkunskaper inom programmering. Det passar perfekt dem som vill börja lära 
sig programmering på rätt sätt som kommer täcka de mesta grunderna där en ny programmerare bör tänka på. Språket kommer även passa lärarna som vill lära
ut programmering till de nybörjare eller möjligtvis till en grupp av barn i grundskolan.


\section{Användarhandledning}

\subsection{Installation}
För att kunna testa ETL krävs den senaste versionen av Ruby installerad. 

Användaren behöver skriva kommandoraden \textbf{\textit{ruby ETL.rb}} för att kunna köra programmet.

Det finns två sätt att köra ETL språket på:
\begin{enumerate}
    \item Att skriva kod genom terminalen, vilket är ett sätt om användaren vill skriva endast en enkel rad kod som inte består av flera saker samtidigt.
    Detta kan användaren göra i ETL.rb genom: se \textbf{Figur 1} i sektionen \textbf{Bilder}!.
    \item Andra sättet är att testa språket i sin helhet vilket innebär att användaren skriver sin kod i en fil som heter \textbf{etl.etl} där kommer 
    programmet ta hand om resten. Detta kan användaren göra i ETL.rb genom: se \textbf{Figur 2} i sektionen \textbf{Bilder}!.     
\end{enumerate}  

\newpage
\subsection{Variabler och Tilldelning}
Variabler har en dynamisk typning där användaren behöver inte specificera datatypen när den ska deklareras. Tilldelningen i ETL betecknas endast med 
tilldelningsoperatorn “=”. I ETL går det att tilldela en variabel till booleska värden, strängar och matematiska uttryck.

Det innebär att det ska finnas endast ett namn och dennes värde vilket visas i följande stil:
\begin{verbatim}
    x = 5 
    y = "Hej"
    z = "hej" plus "då"
    d = 5 < 10
\end{verbatim}


\subsection{Matematiska Operationer}
ETL kan utföra alla sedvanliga matematiska beräkningar såsom addition, subtraktion, multiplikation och division samt deras rätta prioriteter 
och associativiteten det vill säga division och multiplikation ska utföras före addition och subtraktion. Samtliga beräkningar utförs oavsett 
de är heltal eller flyttal. Språket stöder även beräkningarna inuti en parentes. Ex:

\begin{verbatim}
    (5 + 4)
    1 - 5
    2 * 1.0
    5 / 5
    4 - 7 * (10 / 2)
\end{verbatim}

Det går även att utföra matematiska beräkningar på variabler som har heltal eller flyttal som värde. Ex:
\begin{verbatim}
    x = 5
    y = x + 2
    z = x * y
\end{verbatim}

\subsection{Kommentarer}
I språket finns det möjligheten att ignorera en rad eller flera rader ifall användaren inte vill att de raderna ska köras. 
Detta görs genom att skriva \textbf{"< <"} för att ignorera en rad och för att ignorera fler rader måste det skrivas 
\textbf{"<comment"} i början av raden och \textbf{"<end"} i slutet av raden.

Exempel på flerradskommentar:
\begin{verbatim}
    <comment 
    Detta är en flerradskommentar och allt som skrivs i det här utrymmet kommer ignoreras och inte köras. 
    Som det syns här går det att skriva vad som helst. ?!”#€%&123456789
    Det är jätteviktigt att inte glömma skriva <end i slutet av raden.
    <end
\end{verbatim}
 
Exempel på enkelradskommentar:
\begin{verbatim}
    << Här ignoreras bara en rad som skrevs med << i början av raden.
\end{verbatim}
 
Kommenterar används ofta av programmerare som en påminnelse på hur dem har kommit fram till den specifika koden.
 


\subsection{Print}

\subsection{Villkor/if-satser}

\subsection{Iterationer}

\subsection{Funktioner}

\subsection{multipleSstrings}



\section{Systemdokumentation}

\subsection{Lexikaliska Analys}

\subsection{Parsning}



\section{Reflektion}




\section{BNF Grammatik}


\newpage
\section{Bilder}

\begin{figure}[h]
\caption{Exempel på hur ska det se ut när användaren vill testa språket genom terminalen}
\centering
\includegraphics[width=\textwidth]{activate_terminal}
\end{figure}

\begin{figure}[h]
\caption{Exempel på hur ska det se ut när användaren vill testa språket genom en test fil}
\centering
\includegraphics[width=\textwidth]{activate_file}
\end{figure}


\end{document}
